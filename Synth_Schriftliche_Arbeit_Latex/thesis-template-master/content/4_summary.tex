\chapter{Fazit}\label{ch:fazit}
Mit den in der Studienarbeit vorgestellten Konzepten, Koroutine, Thread und Asyncio werden sehr nützliche Verfahren angeboten. Sie ermöglichen dem Python-Programmierer parallelartig arbeitende Programme zu entwickelen, um sich von der starren Struktur des sequentiellen Ablaufes zu lösen.
Mit den aufgeführten Beispielen konnte der Ablauf und auch die zu implementierende Struktur der Konzepte aufgezeigt werden. Neben den Threads, welche im Allgemeinen recht bekannt und verbreitet sind, erweisen sich gerade die Koroutine bzw. Asyncio als sehr nützlich. 
Besonders die Verwendung des kooperativen Multitaskings macht diese Verfahren robust gegenüber Zugriffsfehlern, welche bei Threads ohne Synchronisation sehr leicht auftreten können. 
Der zeitliche Vorteil, der beim Einsatz dieser Pakete entstehen kann, ist bei Asyncio deutlich sichtbar und auch leicht nachvollziebar. Vorsicht und bedachtes Vorgehen sind jedoch geboten, da leicht unübersichtliche Situationen entstehen können.

