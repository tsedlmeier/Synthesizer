\chapter{Einleitung}
\label{ch:intro}
Die Welt der Technik befindet sich seit ihrem Beginn im ständigen Wandel. Neben immer komplexer und kleiner werdender Hardware steigen auch die Anforderung stetig bezüglich leistungsfähiger und flexibler Software. Besonders die Verarbeitungsgeschwindigkeit und Reaktionsfähigkeit von Software-Systemen rückten dabei zunehmend in den Fokus. So haben in den Anfangszeiten der Informatik gerade sequentiel oder seriel abarbeitende Programme Anwendung gefunden. Dies bedeutet, dass jede Funktion nacheinander Zeile für Zeile abgearbeitet wurde. Die Abarbeitung entspricht einem linearen Vorgehen. Für viele Anwendungen ist dieses Verhalten jedoch gänzlich ungeeignet. Somit etablierte sich neben der sequentiellen Abarbeitung zunehmend die parallele oder quasiparallele Durchführung. Gerade bei den heutigen Mikrocontroller- oder Prozessor-Familien werden mit geschickten Scheduling-Mechanismen die vielfälltig einsetzbaren eingebetten Systeme, die beinahe in jedem Bereich des Altags vorgedrungen sind, erst möglich.\\
In Python sind diese Werkzeuge in Form von Koroutinen, Threads oder Asnycio auf Programmierebene gegeben. Sinnvoll und bedacht eingesetzt, können sie merkliche Geschwindigkeits- und Flexibilitätsvorteile für den Anwender erbringen.
Neben der grundlegenden theoretischen Erläuterung der verschiedenen Konzepte sollen durch die nähere Betrachtung eines exemplarischen Anwendungsfalls die Zusammenhänge besser verdeutlicht werden.
