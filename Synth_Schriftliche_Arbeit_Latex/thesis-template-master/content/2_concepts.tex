\chapter{Konzepte}
\label{ch:concept}

\section{Zielsetzung}
\label{sec:zielsetzung}
Wie bereits in Kapitel \ref{ch:intro} beschrieben, dient diese Projektarbeit zur Wissenserweiterung im Bereich der analogen Schaltungstechnik.
Darüber hinaus soll im Zuge dieser Arbeit ein einsetzbarer modularer Synthesizer gebaut werden, der zu elektronischen Klangerzeugung genutzt werden kann. 
Der Synthesizer soll aus verschiedenen Modulen bestehen, welche unabhängig von einander genutzt werden können. 
Der weitere Aufbau wird in Abschnitt \ref{sec:AufbauSynth} genauer beschrieben. 
Darüber hinaus werden in Abschnitt \ref{sec:AnalogePrinzipien} grundlegende Prinzipien erläutert, die insbesondere bei der elektronischen Klangerzeugung Anwendung finden.


\section{Aufbau eines modularen Synthesizers}
\label{sec:AufbauSynth}
Wie bereits in Abschnitt \ref{sec:zielsetzung} erläutert, besteht ein modularer Synthesizer aus mehreren vereinzelten Modulen. 
Diese Module können mit Kabeln verbunden und somit in Interaktion miteinander gebracht werden. 

Um eine grundlegende Funktion zu ermöglichen, ist ein Basisumfang an Modulen nötig.
Die hierfür nötigen Komponenten oder Module werden im Folgenden aufgelistet und kurz erläutert.

\begin{itemize}
	\item Netzteil:\newline
	Das Netzteil ist elementarer Bestandteil des Synthesizers und stellt die benötigten Spannungslevel zur Versorgung der einzelnen Module bereit.
	Insbesondere für den Einsatz von Operationsverstärkern ist es nötig symmetrische Spannungsversorgungen bereit zu stellen.
	
	\item LFO: \newline
	Ein LFO ("Low Frequency Oscillator") wird genutzt, um niederfrequente Signale zu erzeugen.
	Typischerweise wird dieses Modul genutzt, um andere Module anzusteuern.
	
	\item VCO: \newline
	Der VCO ist ein spannungsgesteuerter Oszillator und stellt die Basis bei analogen Synthesizern dar.
	Über eine Steuerspannung kann die Frequenz des erzeugten Signals und somit die Tonhöhe verändert werden. 
	Verbreitete Signale zur elektronischen Tonerzeugung stellen das Sägezahn- und das Rechtecksignal dar.
	
	\item Sequenzer: \newline
	Der Sequenzer erzeugt seriell alternierende Spannungsfolgen, die durch verschiedene Kippschalter und Potentiometer sowohl die einzelnen Spannungspegel als auch die gesamte Geschwindigkeit des Signals variieren. In der Regel werden die Ausgangssignale des Sequenzers zur Ansteuerung weiterer Module – den sogenannten Spannungsgesteuerten-Modulen – hergenommen. Neben den Oszillatoren bildet der Sequenzer somit die Basis der Synthesizer-Module.
	\item Filter
	\item Mischer
	\item Gehäuse : \newline
	Um den Synthesizer gut bedienen zu können und um die enthaltenen Komponenten vor schädlichen Einflüssen zu Schützen ist es sinnvoll, 
	die Module in einem Gehäuse zu verbauen. Dieses besteht üblicherweise aus zwei Schienen mit Anschraubmöglichkeiten, 
	auf welchen die Frontplatten der einzelnen Module geschraubt werden können.  
\end{itemize}

Um den groben Aufbau und die dahinter liegende Struktur zu verdeutlichen, ist in Abbildung xxx die grobe Produktarchitektur aufgezeigt.


\begin{figure}[h]
	\centering
	\setlength{\fboxsep}{1pt} %Abstand der Linien zur Abbildung
	\setlength{\fboxrule}{1pt} %Dicke der Linie
	\fbox{\includegraphics[width=0.8\textwidth]{figures/Produktarchitektur.pdf}}
	\caption{Produktarchitektur des modularen Synthesizers}
	\label{fig:fig:Produktarchitektur}
\end{figure}


\newpage
\section{Grundlegende analoge Prinzipien}
\label{sec:AnalogePrinzipien}

Exkurs zum Thema Ton/Klang 
Grundton, Obertöne etc.!




