\chapter{Einleitung}
\label{ch:intro}

Die Welt der Elektrotechnik ist mehr denn je im Umbruch und Wandel. Um für diese Herausforderungen gerüstet zu sein, 
ist es fundamental mit den Grundlagen dieses breitgefächerten Themengebietes bestens vertraut zu sein.
Besonders im Bereich der analogen Schaltungstechnik ist es jedoch oft schwierig ein 
tieferes Verständnis für Vorgänge in komplexen Aufbauten zu erlangen. Die Mathematik bietet zwar meist sehr akkurate Mittel, um eine Schaltung ausreichend zu beschreiben, oft reicht dies jedoch für Anfänger nicht aus, um das Verhalten greifbar zu machen.
Abhilfe kann hier die Visualisierung oder Simulation der entsprechenden Spannungsverläufe schaffen. 
Neben der visuellen Analyse kann jedoch auch der Klang von Signalverläufen tieferes Verständnis aufbauen.
Durch die Durchführung dieser Projektarbeit soll besonders dieser Aspekt vertieft werden 
und somit der Wissensstand bezüglich analoger Schaltungstechnik im Allgemeinen ausgebaut werden. 
Darüber hinaus soll Elektrotechnik durch die elektronische Klangerzeugung für Außenstehende besser erfahrbar gemacht werden.
