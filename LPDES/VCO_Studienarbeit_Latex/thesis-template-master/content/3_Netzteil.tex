\chapter{Netzteil}
\label{ch:Netzteil}

\section{Allgemeines}
Um eine fehlerfreie Funktion aller weiteren Module zu gewährleisten, ist es wichtig, dass eine geeignete Spannungsversorgung bereitgestellt wird. 
Insbesondere für die Opamp-Schaltungen ist es wichtig, dass eine störfreie symmetrische Spannung bereitgestellt wird. 
Im Bereich der modularen Synthesizer wird hier typischerweise eine Spannung von +/- 12 V benötigt. Diese Spannungspegel werden durch eine geignete Beschaltung von Dioden und Kondensatoren generiert, welche in Abschnitt \ref{sec:netzteil_schaltplan} bzw. \ref{sec:netzteil_umsetzung} genauer erläutert wird. Hierbei wird die für modulare Synthesizer typischerweise eingesetzte Wannenstecker-Belegung verwendet. Diese wird entsprechend in allen Modulen eingesetzt!


\section{Schaltplan}
\label{sec:netzteil_schaltplan}
bla bla bla

\section{Umsetzung}
\label{sec:netzteil_umsetzung}
bla bla bla